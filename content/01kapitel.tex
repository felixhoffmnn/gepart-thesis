%!TEX root = ../dokumentation.tex

\chapter{Einleitung} \label{ch:introduction}

Social-Media-Plattformen, Medienhäuser und Politiker sind maßgeblich daran beteiligt, über Ereignisse zu berichten und diese von ihrer Perspektive wiederzugeben und zu bewerten. Nach \textcite{willeke_soziale_2019} neigen Politiker dazu, mit ihrem eigenen Stil -- semantisch, syntaktisch und pragmatisch -- ihre Wählerschaft zu beeinflussen. Nicht nur Politiker, sondern auch politisch interessierte Personen ohne Parteiamt nutzen die Möglichkeit der niedrigschwelligen Kommunikation und Diskussion in sozialen Medien, um ihre Meinung schnell und direkt zu äußern. Dabei ist häufig unklar, welchen politischen Hintergrund eine Person oder auch ein Medium wie eine Zeitung hat.

In dieser Arbeit werden \ac{NLP}-Methoden zur Analyse und Klassifikation politischer Texte untersucht und bewertet. Ein Klassifikationsmodell, das überprüft, ob ein Text eine starke Zuneigung zu einzelnen Parteien des Deutschen Bundestages aufweist, könnte dazu genutzt werden, um Zeitungsartikel zu klassifizieren. Das würde Lesern mehr Transparenz über politische Zuneigungen von Medienhäusern und anderweitigen Texten geben.

\section{Textanalyse in politischer Kommunikation \& Parteienforschung} \label{sec:introductionTextanalysis}

Die folgenden Literaturhinweise bieten einen Überblick über verschiedene Methoden und Ansätze zur Text-Klassifikation im politischen Kontext, insbesondere im Hinblick auf die Analyse von Twitter-Texten von Politikern.

\textcite{saltzer_bundestagswahl_2022} betrachten die Positionen von Bundestagskandidaten auf Twitter und betrachtet diese einerseits im Vergleich innerhalb der Parteien (Flügel/Strömungen) sowie andererseits auf einem allgemeinen politischen Koordinatensystem \autocite{saltzer_bundestagswahl_2022, saltzer_finding_2022}. Für die Analyse verwendete \textcite{saltzer_finding_2022} unter anderem Tweet-Texte, etwaige Zugehörigkeit zu einer Vorfeldorganisation, Merkmale des Politikers, als auch ihr Abstimmungsverhalten.

\textcite{li_survey_2021} sowie \textcite{kowsari_text_2019} bieten je einen umfassenden Überblick sowie Vor- und Nachteile verschiedener Arten von Text-Klassifikation und gehen dabei sowohl auf traditionelle als auch auf Deep-Learning-Ansätze ein.
\textcite{minaee_deep_2022} untersuchen und vergleichen die Verwendung von Deep-Learning-Modellen für die Aufgabe der Text-Klassifikation.

\textcite{wong_quantifying_2016} bestimmen die politische Ausrichtung aufgrund des Verhaltens von Personen auf Twitter durch die Betrachtung, welche Accounts ähnliche andere Accounts retweeten.
Zudem vergleichen \textcite{doan_using_2022} verschiedene Sprachmodelle zur Klassifizierung von Reden nach Parteien und führen das für verschiedene Länder bzw. Sprachen durch.
Auch \textcite{biessmann_predicting_2016} klassifizieren Reden nach Parteien und nutzen zum Trainieren Parlaments-Debatten des Bundestages. Zudem wenden sie den Klassifikator auf andere Arten von Texten wie Social Media Posts an.

Aus den Quellen geht hervor, dass sich bisherige Arbeiten meist ausschließlich mit englischsprachigen Daten aus den USA oder Großbritannien beschäftigt. Außerdem umfassen bisherige Untersuchungen meist einen einzigen Datensatz für politische und sprachliche Analysen.

\section{Ziel dieser Arbeit} \label{sec:thesisGoal}

Ziel dieser Arbeit ist es, mittels \ac{NLP} und \ac{ML} eine Methode zu entwickeln, mit der es möglich ist, Texte von Politikern nach Parteizugehörigkeit zu klassifizieren. Ein solches Modell könnte im Anschluss genutzt werden, um auch Texte ohne gegebene Zuordnung zu klassifizieren.

Im Vergleich zu bisherigen Arbeiten sollen auch Daten aus verschiedenen Themenbereichen und Textsorten\footnote{Texte, die mittels regelbasierten Verfahren klassifiziert werden können (z. B. Gespräch, Erzählung, Werbespruch).} einbezogen werden. Dadurch soll die Klassifikation möglichst losgelöst von Eigenheiten einer konkreten Textart und primär basierend auf sprachlichen und inhaltlichen Unterschieden zwischen den Parteien funktionieren, um folglich auf sämtlichen politischen Texten eingesetzt werden zu können.

Der Name dieser Arbeit, \textit{GePart} (für \enquote{German Party Classification}), steht mit der Anlehnung an das Tier Gepard -- das schnellste Landtier der Erde -- für den Anspruch, dass das resultierende Modell möglichst schnell trainierbar und nutzbar sein soll.

\section{Geplantes Vorgehen}

\ac{CRISP-DM} ist ein weitverbreitetes iteratives Modell, das zur Strukturierung von Data-Mining Projekten genutzt wird \autocite{martinez-plumed_casp-dm_2017, chapman_crisp-dm_2000}. Das Modell besteht aus sechs Schritten: Business Understanding, Data Understanding, Data Preparation, Modeling, Evaluation und Deployment. Die folgende Arbeit wird auf Basis von \ac{CRISP-DM} strukturiert.

In \autoref{ch:crispDm_1} werden zunächst Datensätze ausgewählt, gesichtet und bereinigt. Mit den aufbereiteten Datensätzen werden anschließend in \autoref{ch:crispDm_2} verschiedenartige Klassifikationsverfahren -- klassische \ac{ML}-Verfahren sowie Deep Learning Ansätze -- angewandt. In \autoref{ch:crispDm_3} werden die unterschiedlichen Schritte von dieser Arbeit evaluiert. Außerdem werden weitere Experimente durchgeführt, die an offene Fragestellungen in den durchgeführten Schritten anknüpfen. Schlussendlich wird in \autoref{sec:crispDm_4} das akkurateste Modell öffentlich zur Verfügung gestellt.
