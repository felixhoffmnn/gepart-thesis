%!TEX root = ../dokumentation.tex

\addchap{\langabkverz}
%nur verwendete Akronyme werden letztlich im Abkürzungsverzeichnis des Dokuments angezeigt
%Verwendung:
%		\ac{Abk.}   --> fügt die Abkürzung ein, beim ersten Aufruf wird zusätzlich automatisch die ausgeschriebene Version davor eingefügt bzw. in einer Fußnote (hierfür muss in header.tex \usepackage[printonlyused,footnote]{acronym} stehen) dargestellt
%		\acs{Abk.}   -->  fügt die Abkürzung ein
%		\acf{Abk.}   --> fügt die Abkürzung UND die Erklärung ein
%		\acl{Abk.}   --> fügt nur die Erklärung ein
%		\acp{Abk.}  --> gibt Plural aus (angefügtes 's'); das zusätzliche 'p' funktioniert auch bei obigen Befehlen
%	siehe auch: http://golatex.de/wiki/%5Cacronym

\begin{acronym}[XXXXXXXX]
    \doublespacing
    \setlength{\itemsep}{-\parsep}
    \acro{AfD}{Alternative für Deutschland}
    \acro{BERT}{Bidirectional Encoder Representations from Transformers}
    \acro{BoF}{Bag-of-Features}
    \acro{BoW}{Bag-of-Words}
    \acro{CDU}{Christlich Demokratische Union Deutschlands}
    \acro{CNN}{Convolutional Neural Network}
    \acro{CPU}{Central Processing Unit}
    \acro{CRISP-DM}{Cross Industry Standard Process for Data Mining}
    \acro{CSU}{Christlich-Soziale Union in Bayern e. V.}
    \acro{FDP}{Freie Demokratische Partei}
    \acro{GDDR5}{Graphics Double Data Rate 5}
    \acro{GloVe}{Global Vectors for Word Representation}
    \acro{GPU}{Graphics Processing Unit}
    \acro{LSTM}{Long Short-Term Memory}
    \acro{MdB}{Mitglied des Deutschen Bundestages}
    \acro{ML}{Machine Learning}
    \acro{MLP}{Multi-Layer Perceptron}
    \acro{NB}{Naive Bayes}
    \acro{NLP}{Natural Language Processing}
    \acro{PDF}{Portable Document Format}
    \acro{RAM}{Random-Access Memory}
    \acro{ReLU}{Rectified Linear Unit}
    \acro{SPD}{Sozialdemokratische Partei Deutschlands}
    \acro{SVC}{Support Vector Classification}
    \acro{SVM}{Support Vector Machine}
    \acro{TF-IDF}{Term Frequency -- Inverse Document Frequency}
    \acro{URL}{Uniform Resource Locator}
\end{acronym}
