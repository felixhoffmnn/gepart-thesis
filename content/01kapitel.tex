%!TEX root = ../dokumentation.tex

\chapter{Einleitung} \label{ch:introduction}

Social-Media-Plattformen, Medienhäuser und Politiker sind maßgeblich daran beteiligt, über Ereignisse zu berichten, diese aus ihrer Perspektive darzustellen und zu bewerten. Nach \textcite{willeke_soziale_2019} neigen Politiker dazu, ihre Wählerschaft durch ihren eigenen Stil -- semantisch, syntaktisch und pragmatisch -- zu beeinflussen. Nicht nur Politiker, sondern auch politisch interessierte Personen ohne Parteiamt nutzen die Möglichkeit der niedrigschwelligen Kommunikation und Diskussion in sozialen Medien, um ihre Meinung schnell und direkt zu äußern. Dabei bleibt häufig unklar, welchen politischen Hintergrund eine Person oder ein Medium wie eine Zeitung hat.

In dieser Arbeit werden \ac{NLP}-Methoden zur Analyse und Klassifikation politischer Texte untersucht und bewertet. Ein Klassifikationsmodell, das überprüft, ob ein Text eine starke Zuneigung zu einzelnen Parteien des Deutschen Bundestages aufweist, könnte beispielsweise dazu genutzt werden, um Zeitungsartikel zu klassifizieren. Das würde Lesern mehr Transparenz hinsichtlich der politischen Zuneigungen von Medienhäusern und anderen Texten geben.

\section{Textanalyse in politischer Kommunikation \& Parteienforschung} \label{sec:introductionTextanalysis}

Die folgenden Literaturhinweise bieten einen Überblick über verschiedene Methoden und Ansätze zur Textklassifikation im politischen Kontext, insbesondere im Hinblick auf die Analyse von Twitter-Texten von Politikerinnen und Politikern.

\textcite{saltzer_bundestagswahl_2022} betrachten die Positionen von Bundestagskandidatinnen und -kandidaten auf Twitter, sowohl im Vergleich innerhalb der Parteien (Flügel/Strömungen) als auch im allgemeinen politischen Koordinatensystem \autocite{saltzer_bundestagswahl_2022, saltzer_finding_2022}. Für die Analyse verwendet \textcite{saltzer_finding_2022} unter anderem Tweet-Texte, eine etwaige Zugehörigkeit zu einer Vorfeldorganisation, Merkmale des Politikers oder der Politikerin sowie deren Abstimmungsverhalten.

\textcite{li_survey_2021} sowie \textcite{kowsari_text_2019} bieten jeweils einen umfassenden Überblick sowie Vor- und Nachteile verschiedener Arten von Textklassifikation und gehen dabei sowohl auf traditionelle als auch auf Deep-Learning-Ansätze ein.
\textcite{minaee_deep_2022} untersuchen und vergleichen die Verwendung von Deep-Learning-Modellen für die Aufgabe der Textklassifikation.

\textcite{wong_quantifying_2016} bestimmen die politische Ausrichtung aufgrund des Verhaltens von Personen auf Twitter, indem sie betrachten, welche Accounts ähnliche andere Accounts retweeten.
Zudem vergleichen \textcite{doan_using_2022} verschiedene Sprachmodelle zur Klassifizierung von Reden nach Parteien und führen dies für verschiedene Länder bzw. Sprachen durch.
Auch \textcite{biessmann_predicting_2016} klassifizieren Reden nach Parteien und nutzen zum Trainieren Parlamentsdebatten des Bundestages. Sie wenden den Klassifikator zudem auf andere Arten von Texten wie Social-Media-Posts an.

Aus den Quellen geht hervor, dass bisherige Arbeiten sich meist ausschließlich mit englischsprachigen Daten aus den USA oder Großbritannien beschäftigt haben. Zudem umfassen bisherige Untersuchungen meist nur einen einzigen Datensatz für politische und sprachliche Analysen.

\section{Ziel dieser Arbeit} \label{sec:thesisGoal}

Ziel dieser Arbeit besteht darin, mittels \ac{NLP} und \ac{ML} ein Modell zu entwickeln, das es ermöglicht, Texte von Politikern nach Parteizugehörigkeit zu klassifizieren. Ein solches Modell kann anschließend genutzt werden, um auch Texte ohne gegebene Zuordnung zu klassifizieren.

Im Vergleich zu bisherigen Arbeiten werden auch Daten aus verschiedenen Themenbereichen und Textsorten\footnote{Texte, die mittels regelbasierten Verfahren klassifiziert werden können (z. B. Gespräch, Erzählung, Werbespruch).} einbezogen. Dadurch funktioniert die Klassifikation möglichst losgelöst von Eigenheiten einer konkreten Textart und primär basiert sie auf sprachlichen und inhaltlichen Unterschieden zwischen den Parteien, um auf sämtlichen politischen Texten eingesetzt werden zu können.

Der Name dieser Arbeit, \textit{GePart} (für \enquote{German Party Classification}), steht in Anlehnung an das Tier Gepard -- das schnellste Landtier der Erde -- für den Anspruch, dass das resultierende Modell schnell trainierbar und nutzbar sein soll.

\section{Geplantes Vorgehen}

Der \ac{CRISP-DM} ist ein weitverbreitetes iteratives Modell, das zur Strukturierung von Data-Mining-Projekten genutzt wird \autocite{martinez-plumed_casp-dm_2017, chapman_crisp-dm_2000}. Das Modell besteht aus sechs Schritten: Business Understanding, Data Understanding, Data Preparation, Modeling, Evaluation und Deployment. Diese Arbeit wird auf Basis des \ac{CRISP-DM}-Modells strukturiert.

In \autoref{ch:crispDm_1} werden zunächst Datensätze ausgewählt, gesichtet und bereinigt. Mit den aufbereiteten Datensätzen werden in \autoref{ch:crispDm_2} verschiedenartige Klassifikationsverfahren -- klassische \ac{ML}-Verfahren sowie Deep-Learning-Ansätze -- angewandt. In \autoref{ch:crispDm_3} werden die unterschiedlichen Schritte dieser Arbeit evaluiert. Außerdem werden weitere Experimente durchgeführt, die an offene Fragestellungen in den durchgeführten Schritten anknüpfen. Schlussendlich wird in \autoref{sec:crispDm_4} das akkurateste Modell öffentlich zur Verfügung gestellt.
