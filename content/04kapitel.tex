%!TEX root = ../dokumentation.tex

\chapter{Evaluierung eines Klassifikationsmodells} \label{ch:crispDm_2}

\section{Evaluation} \label{sec:evaluation}

% TODO: Test specific domain model with other data (aplly tweets to speeches model)

\begin{table}[H]
    \centering
    {\footnotesize
    \begin{tblr}{width=\textwidth, hlines, vlines}
        \textbf{Datensatz} & Tweets & Wahlpro\-gramme & Reden \\ 

        Tweets & N/A & \num{0} & \num{0} \\
        Wahlpro\-gramme & \num{0} & N/A & \num{0} \\
        Reden & \num{0} & \num{0} & N/A \\
    \end{tblr}
    }
    \caption{Performance von domainspezifischen Modellen auf alternativen Testdaten} \label{tab:comparisonModelDatasets}
\end{table}

% TODO: Update scores and maybe add more models

\begin{table}[H]
    \centering
    {\footnotesize
    \begin{tblr}{width=\textwidth, hlines, vlines, colspec={l*{3}{Q[si={table-format=1.2},c]}}, row{1}={guard,font=\bfseries,l}, row{5}={font=\bfseries}}
        Datensatz & Precision & Recall & \(F_1\) Score \\ 

        Tweets & 0.59 & 0.59 & 0.59 \\
        Wahlpro\-gramme & 0 & 0 & 0 \\
        Reden & 0 & 0 & 0 \\

        Kombiniert & 0 & 0 & 0 \\
    \end{tblr}
    }
    \caption{Vergleich des \(F_1\) Scores zwischen \texttt{fasttext} und \acs{BERT}} \label{tab:comparisonModels}
\end{table}

\section{Deployment} \label{sec:deployment}

\subsection{Interface}

\begin{itemize}
    \item Streamlit
    \item Huggingface
\end{itemize}

\subsection{Beispielanwendung: Zeitungen}

\begin{itemize}
    \item Modell auf Zeitungsartikel zwischen 2017 und 2021 anwenden (Bild, Zeit, ...)
    \item Modell auf Tweets von Zeitungen anwenden
\end{itemize}
    