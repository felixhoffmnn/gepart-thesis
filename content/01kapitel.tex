%!TEX root = ../dokumentation.tex

\chapter{Introduction} \label{ch:introduction}

% TODO: Move metion of GitHub

Der Quellcode für diese Arbeit ist verfügbar unter \href{https://github.com/felixhoffmnn/studienarbeit}{https://github.com/felixhoffmnn/studienarbeit}.

Ereignisse wie die Wahl von Donald Trump zum US-Präsidenten, das Brexit-Referendum, als auch der Krieg in der Ukraine werfen die Frage auf, welchen Einfluss die Neuen Medien auf die politische Meinungsbildung haben \autocite{brandon_russia_2022, lee_how_2020}. Plattformen wie Twitter, Facebook und Instagram haben in den vergangenen Jahren die Art und Weise, wie Nachrichten erstellt und verbreitet werden, stark verändert. Unter anderem die Möglichkeit eines Nutzers, seine Meinung zu Themen schnell und direkt zu äußern, rückt den konkreten sprachlichen Ausdruck sowie polarisierende Meinungen von Politikern deutlich stärker in den Fokus.

Bisherige Arbeiten beschäftigen sich meist mit englischsprachigen Daten aus den USA oder Großbritannien. Außerdem umfassen Arbeiten wie die von Sältzer und Stier über die Bundestagswahl 2021 lediglich Tweets von Twitter \autocite{saltzer_bundestagswahl_2022}. Dennoch wird in der Arbeit von Sältzer und Stier gezeigt, dass es möglich ist, Trends zu analysieren und die Parteizugehörigkeit eines Politikers anhand seiner Tweets zu klassifizieren.

Ziel dieser Arbeit ist es, mittels Natural Language Processing und Machine Learning die Meinung der Bevölkerung zu einem einzelnen Politiker oder Parteien in Deutschland zu analysieren und ein besseres Verständnis in die Reaktionen und Meinungen der Bevölkerung zu bekommen. Dafür sollen im Vergleich zu bisherigen Arbeiten Daten aus mehreren Quellen einbezogen werden.

Mögliche Fragestellungen sind für diese Arbeit sind:

\begin{itemize}
    \item Lassen sich Trends in der Semantik einzelner Politiker feststellen?
    \item Ist es möglich, auf Basis der Semantik festzustellen, ob ein Politiker zum Flügel einer Partei gehört?
    \item Können Politiker anhand ihrer Nachrichten und Daten klassifiziert werden?
    \item Wie stark ist die Übereinstimmung der Klassifikation mit der wahrhaftigen Parteizugehörigkeit?
    \item Unterstützen Nutzer ebenfalls die Politiker, welche am nächsten an ihrer Meinung sind?
    \item Lässt sich die Parteizugehörigkeit aufgrund einzelner Worte oder Phrasen bestimmen?
\end{itemize}

\section{Ziel dieser Arbeit} \label{sec:thesisGoal}

- Transparenz schaffen
- Analysieren von Zeitungen zur Feststellung von Political Bias
